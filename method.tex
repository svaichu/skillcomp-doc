
A task or a trajectory that a robot has to execute can be considered a sequence of skills. A skill is a basic environment-agnostic action that a robot can execute. For example, a skill could be to pick-kitchen-knife-from-holder robust of parameter values like kitchen counter-top height, knife handle shape, etc.  
We propose creation of a marketplace of skills, from which a robot instance could draw capabilities to execute a task infront it. Onboard robot, a planner strings together skills as needed. 

\subsection{Skill}
A skill is a small action broken from a larger task, that ends with a "neutral state" from which another skill can continue.
Instead of collection of waypoints that falls apart if the objects in the environment are moved, a skill is tolerant to minor changes and transferable to different nature of tasks. 

At implementation, a skill is either:
\paragraph{control logic:} controller utilizing Model Predictive control or PID for simple movements.
\paragraph{neural network model:} trained with Reinforcement Learning (RL), or other methods like Evolutional Algorithims(EA).

During execution, the above model or controller would have access to the robot's sensors. A main problem we would like to address is the ability to transfer a skill learned in one environment to another. Since the action is kept simple, this transfer should be easier than in the case of a end-to-end large model.

Since one skill's training is independent of another, it is possible to experiment different methods for training each skill.
This allows us to exploit domain-specific heuristics, allowing us to trim vast swathes of search space and reduces training complexity for RL based models.
 
\subsection{Planner}
Planner is responsible for odering skill models and executing them. Tasks can be predefined or generative in nature.

\paragraph{Predefined Task:} Often the order of skill-models required to accompish a task is readily available like a receipe and doesnt demand "intelligence" at planning level. 

\paragraph{Generative Task:} In this case, the planner has to decide the order of skill-models to execute. This is a more challenging problem and we would like to leave it out for future work.

Also we would like to investgate how to manage errors during execution and preconditions for start of a skill.


\subsection{Marketplace of Skills}

This marketplace is a repository of skills that can be used by robots to execute