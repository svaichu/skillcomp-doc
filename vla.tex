Vision-Language-Action (VLA) models are a class of models based on the transformer architecture and are designed to process and understand both visual and textual information. They are made up of two parts: 

1. VLM (Vision-Language Model): This part of the model is responsible for understanding and processing visual and textual information. 

2. Action Head: Made of diffusion transformers and are responsible for generating actions using the embeddings from the VLM. They can also take in additional inputs such as proprioceptive and previleged environment information.

Some of the popular VLA models include:
- SoFar \cite{qiSoFarLanguageGroundedOrientation2025}, state of the art in robot manipulation tasks in the Google SimplerEnv dataset \cite{liEvaluatingRealWorldRobot2024}.

- OpenVLA \cite{kimOpenVLAOpenSourceVisionLanguageAction2024}, a 7B parameter model. It uses Lama 2 as the backbone and is trained on 970k robot demos in Open X-Embodiment dataset \cite{collaborationOpenXEmbodimentRobotic2024}.

- SAM2ACT \cite{fangSAM2ActIntegratingVisual2025} is a new model focused on manipulation tasks. It is the leader in the RLbench dataset \cite{jamesRLBenchRobotLearning2019a}. 